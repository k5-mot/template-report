\section{\(R^{3}\)における等式拘束条件付き極値問題} \label{section:1}

\subsection{問題}

\(\bigtriangleup{ABC}\) において、各頂点と向かい合う辺の長さをそれぞれ\(a,b,c\)とする。
また、各辺の長さの和を\(2\ell\)
(すなわち、\(a + b + c = 2 \ell\)、\(\ell\)は各辺の長さの和の半分)とする。
さらに、\(\bigtriangleup{ABC}\)の面積\(S\)の\(2\)乗を\(J\)とすると、
\begin{equation}\label{equation:1_1}
  J = S^{2} = \ell \left(\ell - a\right) \left(\ell - b\right) \left(\ell - c\right)
\end{equation}
で与えられる。
以下の問に答えなさい。

\renewcommand{\labelenumi}{(\arabic{enumi})}
\begin{enumerate}
  \item (\ref{equation:1_1})式が成り立つことを示しなさい。
  \item \(\bm{x} = \begin{bmatrix}a\\b\\c\\ \end{bmatrix} \in R^{3}\)とする。
  このとき、三角形の辺の長さの和が一定(\(a + b + c = 2 \ell\)、一定)である三角形のうち、
  その面積(の2乗)が最大となる\(\bm{x}\)を求め、そのときの三角形はどのようなものであるか
  理由を添えて説明しなさい。
\end{enumerate}

\subsection{回答}

\begin{enumerate}
  \item 面積公式より、
  \begin{equation}\label{equation:1_1_a_1}
    S = \frac{1}{2} a b \sin{C}
  \end{equation}
  となる。\(\sin^2 C + \cos^2 C = 1\)を用いて\(\sin\)を\(\cos\)に直すと、
  \begin{equation}\label{equation:1_1_a_2}
    S = \frac{1}{2} a b \sqrt{1 - cos^2 C}
  \end{equation}
  となる。余弦定理より\(\cos C = \frac{a^2 + b^2 - c^2}{2ab}\)なので、
  \begin{equation}\label{equation:1_1_a_3}
    S = \frac{1}{2} a b \sqrt{1 - \left(\frac{a^2 + b^2 - c^2}{2ab}\right)^2}
  \end{equation}
  (\ref{equation:1_1_a_3})式を変形すると、
  \begin{equation}\label{equation:1_1_a_4}
    \begin{split}
      S &= \frac{1}{2} a b \sqrt{1 - \left(\frac{a^2 + b^2 - c^2}{2ab}\right)^2}\\
        &= \frac{1}{2} a b \sqrt{\frac{(2ab)^2 - (a^2 + b^2 - c^2)^2}{(2ab)^2}}\\
        &= \frac{1}{4} \sqrt{(2ab)^2 - (a^2 + b^2 - c^2)^2}\\
        &= \frac{1}{4} \sqrt{(2ab + a^2 + b^2 - c^2) (2ab - a^2 - b^2 + c^2)}\\
        &= \frac{1}{4} \sqrt{((a + b)^2 - c^2) (c^2 - (a - b)^2)}\\
        &= \frac{1}{4} \sqrt{(a + b + c)(a + b - c)(a - b + c)(-a + b + c)}\\
        &= \sqrt{\frac{a + b + c}{2}\cdot\frac{a + b - c}{2}\cdot\frac{a - b + c}{2}\cdot\frac{-a + b + c}{2}}\\
    \end{split}
  \end{equation}
  条件より、\(\ell = \frac{a + b + c}{2}\)なので、
  \begin{equation}\label{equation:1_1_a_5}
    S = \sqrt{\ell \cdot \left(\ell - a\right) \cdot \left(\ell - b\right) \cdot \left(\ell - c\right)}
  \end{equation}
  両辺を2乗すると、
  \begin{equation}\label{equation:1_1_a_6}
    J = S^2 = \ell \cdot \left(\ell - a\right) \cdot \left(\ell - b\right) \cdot \left(\ell - c\right)
  \end{equation}
  よって、(\ref{equation:1_1})式が成り立つことを示せた。
  \begin{figure}[htbp]
    \centering
    \includegraphics[width=60mm]{./image/dai1.png}
    \caption{各頂点と向かい合う辺の長さをそれぞれ\(a,b,c\)とする\(\bigtriangleup{ABC}\)}
    \label{figure:dai1}
  \end{figure}

  % \item \(2 \ell = a + b + c\)より、
  % \begin{equation}\label{equation:1_2_a_1}
  %   c = 2 \ell - \left(a + b\right)
  % \end{equation}
  % であり、(\ref{equation:1_2_a_1})式より、
  % \begin{equation}\label{equation:1_2_a_2}
  %   J = \ell \cdot \left(\ell - a\right) \cdot \left(\ell - b\right) \cdot \left(a + b - \ell\right)
  % \end{equation}
  % ここで、\(\bm{x} = \begin{bmatrix}a\\b\\c\\ \end{bmatrix} \in R^{3}\)に対して、
  % 汎関数
  % \begin{equation}\label{equation:1_2_a_3}
  %   J\left(\bm{x}\right) = \ell \cdot \left(\ell - a\right) \cdot \left(\ell - b\right) \cdot \left(a + b - \ell\right)
  % \end{equation}
  % の方向微分を計算する。\\
  % \(\bm{x} = \begin{bmatrix}a\\b\\c\\ \end{bmatrix} \in R^{3}\)なので、
  % \(\bm{h} = \begin{bmatrix}h_{a}\\h_{b}\\h_{c}\\ \end{bmatrix} \in R^{3}\)と置く。
  % このとき、
  % \begin{equation}\label{equation:1_2_a_4}
  %   \bm{x} + \beta \bm{h} =
  %   \begin{bmatrix}
  %     a + \beta h_{a}\\
  %     b + \beta h_{b}\\
  %     c + \beta h_{c}\\
  %   \end{bmatrix} \in R^{3}
  % \end{equation}
  % とする。\\
  % すなわち、\(a\)は\(a + \beta h_{a}\)と増加し、
  % \(b\)は\(b + \beta h_{b}\)と増加し、
  % \(c\)は\(c + \beta h_{c}\)と増加する。\\
  % したがって、(\ref{equation:1_2_a_2})式より、
  % \begin{equation}
  %   J\left(\bm{x}\right) = \ell \cdot \left(\ell - a - \beta h_{a}\right) \cdot \left(\ell - b - \beta h_{b}\right) \cdot \left(a + \beta h_{a} + b + \beta h_{b}- \ell\right)
  % \end{equation}

  \item ラグランジュの未定乗数\(\lambda\)を導入し、
  汎関数\(L\left(\ell, \lambda\right)\)を定義すると、
  \begin{equation}\label{equation:1_2_a_1}
    L\left(a, b, c, \lambda\right) =
    \ell \left(\ell - a\right) \left(\ell - b\right) \left(\ell - c\right)
    - \lambda \left(a + b + c - 2 \ell\right)
  \end{equation}
  \(L\left(a, b, c, \lambda\right)\)の偏微分をそれぞれ求めて、極値を求める。
  \begin{equation}\label{equation:1_2_a_2}
    \left\{
    \begin{split}
      &\frac{dL}{da} = \ell (-1)    (\ell - b) (\ell - c) - \lambda (-1) = -\ell (\ell - b) (\ell - c) + \lambda = 0\\
      &\frac{dL}{db} = \ell (\ell - a) (-1)    (\ell - c) - \lambda (-1) = -\ell (\ell - a) (\ell - c) + \lambda = 0\\
      &\frac{dL}{dc} = \ell (\ell - a) (\ell - b) (-1)    - \lambda (-1) = -\ell (\ell - a) (\ell - b) + \lambda = 0\\
      &\frac{dL}{d\lambda} = - \left(a + b + c - 2 \ell\right) = 0\\
    \end{split}
    \right.
  \end{equation}
  この連立方程式を解く。
  \begin{equation}\label{equation:1_2_a_3}
    \left\{
    \begin{split}
      \ell (\ell - b) (\ell - c) - \lambda = 0\\
      \ell (\ell - a) (\ell - c) - \lambda = 0\\
      \ell (\ell - a) (\ell - b) - \lambda = 0\\
      a + b + c = 2 \ell
    \end{split}
    \right.
  \end{equation}
  \begin{equation}\label{equation:1_2_a_4}
    \left\{
    \begin{split}
      \ell (\ell - b) (a + b - \ell) - \lambda = 0 \Leftrightarrow (\ell - b) (a + b - \ell) = \frac{\lambda}{\ell} \\
      \ell (\ell - a) (a + b - \ell) - \lambda = 0 \Leftrightarrow (\ell - a) (a + b - \ell) = \frac{\lambda}{\ell} \\
      \ell (\ell - a) (\ell - b) - \lambda = 0 \Leftrightarrow (\ell - a) (\ell - b) = \frac{\lambda}{\ell}
    \end{split}
    \right.
  \end{equation}
  \begin{equation}\label{equation:1_2_a_5}
    \left\{
    \begin{split}
      (\ell - b) (a + b - \ell) = (\ell - a) (\ell - b) \Leftrightarrow a + b - \ell = \ell - a \Leftrightarrow b = 2 \ell - 2a\\
      (\ell - a) (a + b - \ell) = (\ell - a) (\ell - b) \Leftrightarrow a + b - \ell = \ell - b \Leftrightarrow a = 2 \ell - 2b\\
    \end{split}
    \right.
  \end{equation}
  よって、(\ref{equation:1_2_a_5})式より、
  \begin{equation}\label{equation:1_2_a_6}
    \left\{
    \begin{split}
      &a = 2 \ell - 2 \left(2\ell - 2a\right) \Leftrightarrow 3a = 2\ell \Leftrightarrow a = \frac{2}{3} \ell \\
      &b = 2 \ell - 2 \cdot \frac{2}{3} \ell = \frac{2}{3} \ell\\
      &c = 2 \ell - a - b = 2 \ell - \frac{2}{3} \ell - \frac{2}{3} \ell = \frac{2}{3} \ell
    \end{split}
    \right.
  \end{equation}
  % \begin{equation}\label{equation:1_2_a_6}
  %   \begin{split}
  %     a = 2 \ell - 2 \left(2\ell - 2a\right)
  %     & \Leftrightarrow 3a = 2\ell \\
  %     & \Leftrightarrow a = \frac{2}{3} \ell
  %   \end{split}
  % \end{equation}
  % \begin{equation}\label{equation:1_2_a_7}
  %   b = 2 \ell - 2 \cdot \frac{2}{3} \ell = \frac{2}{3} \ell
  % \end{equation}
  % \begin{equation}\label{equation:1_2_a_8}
  %   c = 2 \ell - a - b = 2 \ell - \frac{2}{3} \ell - \frac{2}{3} \ell = \frac{2}{3} \ell
  % \end{equation}
  したがって、面積が最大となる\(a, b, c\)は
  \begin{equation}
    a = b = c = \frac{2}{3} \ell
  \end{equation}
  であり、このときの三角形は、全ての辺が等しい正三角形である。

  % \begin{equation}
  %   \begin{split}
  %     & (\ell - b) (\ell - c) = (\ell - a) (\ell - c) \\
  %     \Leftrightarrow & \ell^2 - (b + c) \ell + bc = \ell^2 - (a + c) \ell + ac \\
  %     \Leftrightarrow & -(b + c) \ell + bc = -(a + c) \ell + ac \\
  %     \Leftrightarrow & (b + c) \ell - bc = (a + c) \ell - ac \\
  %     \Leftrightarrow & (b - a) \ell = (b - a)c \\
  %     \Leftrightarrow & (b - a) \ell = (b - a) (2\ell -a -b) \\
  %     \Leftrightarrow & b = 3\ell - a \\
  %   \end{split}
  % \end{equation}
  % \begin{equation}
  %   \begin{split}
  %     & (\ell - b) (\ell - c) = (\ell - a) (\ell - b) \\
  %     \Leftrightarrow & \ell^2 - (b + c) \ell + bc = \ell^2 - (a + b) \ell + ab \\
  %     \Leftrightarrow & -(b + c) \ell + bc = -(a + b) \ell + ab \\
  %     \Leftrightarrow & (b + c) \ell - bc = (a + b) \ell - ab \\
  %     \Leftrightarrow & (c - a) \ell = (c - a)b \\
  %     \Leftrightarrow & (c - a) \ell = (c - a) (2\ell -a -c) \\
  %     \Leftrightarrow & c = 3\ell - a \\
  %   \end{split}
  % \end{equation}
  % \begin{equation}
  %   \begin{split}
  %     & a + b + c = 2 \ell \\
  %     \Leftrightarrow & a + (3\ell - a) + (3\ell - a) = 2 \ell \\
  %     \Leftrightarrow & -a = -4 \ell \\
  %   \end{split}
  % \end{equation}



  % 相加平均と相乗平均の大小関係より、
  % \begin{equation}\label{equation:1_2_a_1}
  %   \begin{split}
  %     \frac{x + y + z}{3} \geq \sqrt[3]{xyz}\\
  %     \left(x = y = zのとき、等号が成立\right)
  %   \end{split}
  % \end{equation}
  % 両辺を3乗すると、
  % \begin{equation}\label{equation:1_2_a_2}
  %   \left(\frac{x + y + z}{3}\right)^3 \geq xyz
  % \end{equation}
  % (\ref{equation:1_1_a_4})式より、
  % \(x = \ell - a, y = \ell - b, z = \ell - c\)として、
  % (\ref{equation:1_2_a_1})式に代入すると、
  % \begin{equation}\label{equation:1_2_a_3}
  %   \left(\frac{(\ell - a) + (\ell - b) + (\ell - c)}{3}\right)^3 \geq (\ell - a)(\ell - b)(\ell - c)
  % \end{equation}
  % (\ref{equation:1_2_a_2})式の両辺に\(\ell\)を掛ける。
  % \begin{equation}\label{equation:1_2_a_4}
  %   \ell \left(\frac{(\ell - a) + (\ell - b) + (\ell - c)}{3}\right)^3 \geq \ell(\ell - a)(\ell - b)(\ell - c)
  % \end{equation}
  % (\ref{equation:1_1_a_4})、(\ref{equation:1_2_a_2})と
  % \((\ell - a) + (\ell - b) + (\ell - c) = 3 \ell - L = \frac{L}{2}\)より、
  % \begin{equation}\label{equation:1_2_a_5}
  %   S^{2} \leq \frac{L}{2} \left(\frac{L}{6}\right)^{3} = \frac{L^{4}}{432}
  % \end{equation}
  % これより、
  % \begin{equation}
  %   S \leq \frac{L^{2}}{12 \sqrt{3}}
  % \end{equation}
  % 等号は、\(s - a = s - b = s - c\)
  % であり、\(a = b = c\)のとき成立する。\\
  % つまり、三角形が正三角形のとき、三角形の面積が最大になる
\end{enumerate}

\newpage

\section{\(L_{2}\)における等式拘束条件付き極値問題} \label{section:2}

\subsection{問題}

図\ref{figure:2_1}に示すように、\(xy\)平面内で、
一定の長さ\(\ell\)、線密度を\(\rho\)(一定)の伸び縮みしない
一様なひもの両端が点\(A\left(x_{0}, y_{0}\right)\),
\(B\left(x_{1}, y_{1}\right)\)で固定され、
重力の作用で垂れ下がっている。
このときのひもの形状をあらわす曲線\(y\)を求める問題を考える。
以下の問に答えなさい。

\begin{enumerate}
  \item 題意より、与えられたひもが一定の長さ\(\ell\)であることから、
  \begin{equation}\label{equation:2_1}
    G \left(y, y^{\prime}\right) = \int_{x_{0}}^{x_{1}} \sqrt{1 + \left(y^{\prime}\right)^2} dx - \ell = 0
  \end{equation}
  が成り立つことを示しなさい。

  \item ひものポテンシャルエネルギー
  \(F\left(y, y^{\prime}\right)\)は、
  \begin{equation}\label{equation:2_2}
    F\left(y, y^{\prime}\right) = \rho g \int_{x_{0}}^{x_{1}} y \sqrt{1 + \left(y^{\prime}\right)^2} dx
  \end{equation}
  であることを示しなさい。

  \item ひもが重力の作用で垂れ下がって平衡状態にあるときは、
  ひものポテンシャルエネルギーが最小の状態にあることが分かっている(最小原理)。
  そこで、(\ref{equation:2_1})式の\(G \left(y, y^{\prime}\right) = 0\)
  を等式拘束条件をみなし、(\ref{equation:2_2})式の
  \(F\left(y, y^{\prime}\right)\)
  を最小にする\(y \in L_{2} \left[x_{0}, y_{0}\right]\)
  を求めることを考える。
  このとき、(\ref{equation:2_1})式の被積分関数
  \(\sqrt{1 + \left(y^{\prime}\right)^{2}}\)や
  (\ref{equation:2_2})式の被積分関数
  \(y \sqrt{1 + \left(y^{\prime}\right)^{2}}\)が
  \(x\)を陽に含まないことを考慮にいれると、求める\(y\)は
  \begin{equation}\label{equation:2_3_1}
     \int_{x_{0}}^{x_{1}} \sqrt{1 + \left(y^{\prime}\right)^{2}} dx = \ell
    \qquad\quad(拘束条件)
  \end{equation}
  および
  \begin{equation}\label{equation:2_3_2}
    \begin{split}
      \rho g \frac{y}{\sqrt{1 + \left(y^{\prime}\right)^{2}}} - \frac{\lambda}{\sqrt{1 + \left(y^{\prime}\right)^{2}}} = c \\
      (cは定数。\lambda はラグランジュの未定乗数)
    \end{split}
  \end{equation}
  を満足する。(\ref{equation:2_3_2})式が成り立つことを示しなさい。

  \item (\ref{equation:2_3_2})式の解(一般解)を求めるために、媒介変数\(t\)を用いて
  \begin{gather} \label{equation:2_4_1}
    y^{\prime} = \frac{\mathrm{d}y}{\mathrm{d}x} = \sinh t \\ \nonumber
    \left(\sinh は双曲線関数の1つハイパボリックサインを表す\right) \nonumber
  \end{gather}
  と変数変換する。媒介変数による微分の公式を利用することで、
  \begin{equation} \label{equation:2_4_2}
    x = \frac{c}{\rho g} t + c_{1}
    \qquad\quad(c_{1}は任意定数)
  \end{equation}
  が成り立つことを示しなさい。
  さらに、(\ref{equation:2_4_1})式および(\ref{equation:2_3_2})式を用いて、
  求める\(y\)(一般解)は
  \begin{equation} \label{equation:2_4_3}
    y = \frac{\lambda}{\rho g} + \frac{c}{\rho g} \cosh \left(\frac{\rho g}{c} \left(x - c_{1}\right)\right)
    \left(\cosh は双曲線関数の1つハイパボリックコサインを表す\right)
  \end{equation}
  であることを示しなさい。
\end{enumerate}

\begin{figure}[htbp]
  \centering
  \includegraphics[width=80mm]{./image/dai2_figure_a.png}
  \caption{}
  \label{figure:2_1}
\end{figure}

\subsection{回答}

\begin{enumerate}
  \item 点\(AB\)間を微小空間で区切ると、
  微小空間の長さは
  \begin{equation}\label{equation:2_1_a_1}
    ds = \sqrt[]{dx^{2} + dy^{2}} = dx \sqrt[]{1 + \left(\frac{dy}{dx}\right)^{2}}
  \end{equation}
  である。
  全体のひもの長さは区間全体で積分すると求められる。
  \begin{equation}\label{equation:2_1_a_2}
    \begin{split}
      G\left(x, y\right) &= \int^{B}_{A} ds \\
      &= \int^{B}_{A} \sqrt{dx^{2} + dy^{2}}\\
      &= \int^{x_{1}}_{x_{0}} \sqrt[]{1 + \left(\frac{dy}{dx}\right)^{2}} dx\\
      &= \int^{x_{1}}_{x_{0}} \sqrt[]{1 + \left(y^{\prime}\right)^{2}} dx\\
    \end{split}
  \end{equation}
  与えられたひもは伸び縮みせず、長さが\(\ell\)であるので、
  \begin{equation}\label{equation:2_1_a_3}
    G\left(y, y^{\prime}\right) = \int^{x_{1}}_{x_{0}} \sqrt[]{1 + \left(y^{\prime}\right)^{2}} dx - \ell = 0
  \end{equation}
  という等式が成り立つ。
  \begin{figure}[htbp]
    \centering
    \includegraphics[width=50mm]{./image/diff_a.png}
    \caption{}
    \label{figure:2_1_a}
  \end{figure}

  \item 微小空間の質量は、
  \begin{equation}\label{equation:2_2_a_1}
    \begin{split}
      \rho \cdot ds
    \end{split}
  \end{equation}
  微小空間にかかる重力は、重力加速度\(g\)を用いて表すと、
  \begin{equation}\label{equation:2_2_a_2}
    \rho \cdot g \cdot ds
  \end{equation}
  任意の位置\(y\)における微小空間の位置エネルギーは、
  \begin{equation}\label{equation:2_2_a_3}
    \rho \cdot g \cdot y \cdot ds
  \end{equation}
  区間\(A\)から\(B\)まで積分すれば、
  位置エネルギー\(F\left(y, y^{\prime}\right)\)を求められる。
  \begin{equation}\label{equation:2_2_a_4}
    \begin{split}
      F\left(y, y^{\prime}\right) &= \rho g \int^{B}_{A} y ds \\
      &= \rho g \int^{x_1}_{x_0} y \sqrt{1 + \left(y^{\prime}\right)^2} dx
    \end{split}
  \end{equation}

  \item (\ref{equation:2_1_a_3})式、(\ref{equation:2_2_a_4})式より、
  ラグランジュの未定乗数\(\lambda\)を導入し、
  汎関数\(L \left(y, y^{\prime}, \lambda\right)\)を定義すると、
  \begin{equation}
    \begin{split}
      L \left(y, y^{\prime}, \lambda\right)
      &= \rho g \int^{x_1}_{x_0} y \sqrt{1 + \left(y^{\prime}\right)^2} dx -
      \lambda \left(\int^{x_{1}}_{x_{0}} \sqrt{1 + \left(y^{\prime}\right)^{2}} dx - \ell\right) \\
      &= \int^{x_1}_{x_0} \left(\rho g y \sqrt{1 + \left(y^{\prime}\right)^2} -
      \lambda \sqrt{1 + \left(y^{\prime}\right)^{2}}\right) dx + \lambda \ell
    \end{split}
  \end{equation}
  である。\\
  ここで、汎関数\(L \left(y, y^{\prime}, \lambda\right)\)の
  被積分関数\(\hat{L} \left(y, y^{\prime}, \lambda\right)\)を
  \begin{equation}
    \hat{L} \left(y, y^{\prime}, \lambda\right) =
    \rho g y \sqrt{1 + \left(y^{\prime}\right)^2} -
    \lambda \sqrt{1 + \left(y^{\prime}\right)^{2}}
  \end{equation}
  と置く。\\
  \(\hat{L} \left(y, y^{\prime}, \lambda\right)\)は\(x\)を陽に含まないので、
  \begin{equation}
    \hat{L} \left(y, y^{\prime}, \lambda\right) - y^{\prime} \frac{\partial \hat{L} \left(y, y^{\prime}, \lambda\right)}{\partial y^{\prime}} = c
    \quad\quad\left(c は定数\right)
  \end{equation}
  であるという方程式を満たす。
  この方程式を解くと、
  \begin{align}
      & \hat{L} \left(y, y^{\prime}, \lambda\right) - y^{\prime} \frac{\partial \hat{L} \left(y, y^{\prime}, \lambda\right)}{\partial y^{\prime}} = c \notag\\
      \Leftrightarrow & \rho g y \sqrt{1 + \left(y^{\prime}\right)^2} - \lambda \sqrt{1 + \left(y^{\prime}\right)^{2}}
      - \left(\rho g y \frac{\left(y^{\prime}\right)^2}{\sqrt{1 + \left(y^{\prime}\right)^2}} - \lambda \frac{\left(y^{\prime}\right)^2}{\sqrt{1 + \left(y^{\prime}\right)^2}}\right) = c \notag\\
      \Leftrightarrow & \rho g \frac{y}{\sqrt{1 + \left(y^{\prime}\right)^2}} - \frac{\lambda}{\sqrt{1 + \left(y^{\prime}\right)^2}} = c
  \end{align}
  よって、(\ref{equation:2_3_2})式が成り立つことを示せた。
  % オイラー=ラグランジュ方程式を用いて、
  % \begin{equation}
  %   \begin{split}
  %     \frac{\partial L}{\partial x} = \frac{\partial \hat{L}}{\partial y} - \frac{d}{dx} \left(\frac{\partial \hat{L}}{\partial y^{\prime}}\right) = 0
  %   \end{split}
  % \end{equation}
  % を求める。これを変形すると、
  % \begin{equation}\label{equation:2_3_a_8}
  %   \begin{split}
  %     & \frac{\partial \hat{L}}{\partial y} - \frac{d}{dx} \left(\frac{\partial \hat{L}}{\partial y^{\prime}}\right) = 0 \\
  %     \Leftrightarrow & \frac{d}{dx} \left(\frac{\partial \hat{L}}{\partial y^{\prime}}\right) = \frac{\partial \hat{L}}{\partial y} \\
  %     \Leftrightarrow & \frac{d}{dx} \left(\rho g \frac{y y^{\prime}}{\sqrt{1 + \left(y^{\prime}\right)^2}} - \frac{\lambda y^{\prime}}{\sqrt{1 + \left(y^{\prime}\right)^2}}\right) = \rho g \sqrt{1 + \left(y^{\prime}\right)^2} \\
  %     \Leftrightarrow & \rho g \frac{y y^{\prime}}{\sqrt{1 + \left(y^{\prime}\right)^2}} - \frac{\lambda y^{\prime}}{\sqrt{1 + \left(y^{\prime}\right)^2}} = c \\
  %     & \left(c は積分定数\right)\\
  %     \Leftrightarrow & \rho g \frac{y}{\sqrt{1 + \left(y^{\prime}\right)^2}} - \frac{\lambda}{\sqrt{1 + \left(y^{\prime}\right)^2}} = c \\
  %     & \left(c は定数\right)\\
  %   \end{split}
  % \end{equation}
  % となる。
  % よって、(\ref{equation:2_3_2})式が成り立つことを示せた。

  \item 媒介変数による微分の公式を用いて、
  \begin{equation}\label{equation:2_4_a_1}
    \begin{split}
      & \frac{dy}{dx} = \frac{\frac{dy}{dt}}{\frac{dx}{dt}} \\
      \Leftrightarrow & \frac{dx}{dt} = \frac{dx}{dy} \frac{dy}{dt} \\
      \Leftrightarrow & \frac{dx}{dt} = \frac{1}{\frac{dy}{dx}} \frac{dy}{dt} \\
    \end{split}
  \end{equation}
  ここで、
  \begin{equation}\label{equation:2_4_a_2}
    y^{\prime} = \frac{dy}{dx} = \sinh t
  \end{equation}
  と変数変換すると、
  \begin{equation}\label{equation:2_4_a_3}
    \begin{split}
      & \frac{dx}{dt} = \frac{1}{\sinh t} \frac{dy}{dt} \\
    \end{split}
  \end{equation}
  となる。\\
  \(\frac{dy}{dt}\)を求めるために、(\ref{equation:2_3_2})式を変換すると、
  \begin{equation}\label{equation:2_4_a_4}
    \begin{split}
      & \rho g \frac{y}{\sqrt{1 + \left(y^{\prime}\right)^2}} - \frac{\lambda}{\sqrt{1 + \left(y^{\prime}\right)^2}} = c \\
      \Leftrightarrow & y = \frac{\lambda}{\rho g} + \frac{c \sqrt{1 + \left(y^{\prime}\right)^2}}{\rho g} = \frac{\lambda}{\rho g} + \frac{c \sqrt{1 + \left(\sinh t\right)^2}}{\rho g}\\
    \end{split}
  \end{equation}
  となる。これより、\(\frac{dy}{dt}\)は
  \begin{alignat}{2}\label{equation:2_4_a_5}
    \frac{dy}{dt} &= \frac{c}{\rho g} \left(\sqrt{1 + \left(\sinh{t} \right)^2}\right)^{\prime} & \quad \quad \quad & \notag\\
    &= \frac{c}{\rho g} \frac{\left(1 + \left(\sinh{t} \right)^2\right)^{\prime}}{2\sqrt{1 + \left(\sinh{t} \right)^2}} & & \notag\\
    &= \frac{c}{\rho g} \frac{\left(1 + \left(\sinh{t} \right)^2\right)^{\prime}}{2 \sqrt{\cosh^{2}{t}}} & & \left(\because \sinh^{2}{t} + 1 = \cosh^{2}{t}\right) \notag\\
    &= \frac{c}{\rho g} \frac{2 \sinh{t} \left(\sinh{t}\right)^{\prime}}{2 \cosh{t}} & & \notag\\
    &= \frac{c}{\rho g} \frac{2 \sinh{t} \cosh{t}}{2 \cosh{t}} & & \left(\because \frac{d}{dt} \sinh{t} = \cosh{t}\right) \notag\\
    &= \frac{c}{\rho g} \sinh{t} & & \notag \\
  \end{alignat}
  (\ref{equation:2_4_a_3})式と(\ref{equation:2_4_a_5})式を用いて、
  \begin{align}\label{equation:2_4_a_6}
    \frac{dx}{dt} &= \frac{1}{\sinh t} \frac{dy}{dt} \notag\\
    &= \frac{1}{\sinh t} \frac{c}{\rho g} \sinh{t} \notag\\
    &= \frac{c}{\rho g}
  \end{align}
  \(\frac{dx}{dt}\)を不定積分すると、
  \begin{alignat}{2}\label{equation:2_4_a_7}
    x = \frac{c}{\rho g} t + c_{1} & \quad \quad \quad & \left(c_{1}は積分定数\right)
  \end{alignat}
  よって、1つ目の題意を示せた。\\ \\

  (\ref{equation:2_4_a_4})式より、
  \begin{alignat}{2}\label{equation:2_4_a_8}
    y &= \frac{\lambda}{\rho g} + \frac{c}{\rho g}\sqrt{1 + \left(\sinh t\right)^2} & \quad \quad \quad & \notag\\
    &= \frac{\lambda}{\rho g} + \frac{c}{\rho g}\sqrt{\cosh^{2}{t}} & & \left(\because \sinh^{2}{t} + 1 = \cosh^{2}{t}\right) \notag\\
    &= \frac{\lambda}{\rho g} + \frac{c}{\rho g} \cosh{t} & &
  \end{alignat}
  ここで、(\ref{equation:2_4_a_7})式を変形して、
  \begin{equation}\label{equation:2_4_a_9}
    x = \frac{c}{\rho g} t + c_{1} \Leftrightarrow t = \frac{\rho g}{c} \left(x - c_{1}\right)
  \end{equation}
  となり、(\ref{equation:2_4_a_8})式に代入すると、
  \begin{equation}
    y = \frac{\lambda}{\rho g} + \frac{c}{\rho g} \cosh{\left(\frac{\rho g}{c} \left(x - c_{1}\right)\right)}
  \end{equation}
  となる。よって、2つ目の題意を示せた。

\end{enumerate}

\newpage

\section{\(R_{n}\)における等式拘束条件付き極値問題} \label{section:3}

\subsection{問題}

この問題では以下の定理を利用して良い。

\begin{description}
  \item[定理]~\\ \(P\)を\(n \times n\)の対称行列(\(P^{T} = P\)) とし、\(\bm{w} \in R^{n}\)とする。
  \(P\)のすべての固有値が\(0\)以上であるための必要十分条件は、
  すべての零ベクトルでない\(\bm{w} \in R^{n}\)に対して
  \(\bm{w}^{T} P \bm{w} \geq 0\)であることである。
\end{description}

\(A\)を\(n \times n\)行列とし、\(B = A^{T} A\)とする。
\(\bm{x}\)を\(\|\bm{x}\|^{2} = 1\)を満足する\(R_{n}\)に
属するベクトルとし、\(\bm{y} = A \bm{x}\)とする。
このとき、以下の問に答えなさい。

\begin{enumerate}
  \item 行列\(B\)が対称行列であることを示しなさい。
  \item 行列\(B\)のすべての固有値は\(0\)以上であることを示しなさい。
  \item \(\|\bm{x}\|^{2} = 1\)という拘束条件のもとで、
  \(\|\bm{y}\|^{2}\)の最大値は行列\(B\)の
  最大固有値に等しいことを示しなさい。
  そのときの\(\bm{x}\)は\(B\)の最大固有値に対応するノルムが
  \(1\)の固有ベクトルであることを示しなさい。
  また、\(\|\bm{x}\|^{2} = 1\)という拘束条件のもとで、
  \(\|\bm{y}\|^{2}\)の最小値は行列\(B\)の最小固有値に等しいことを示しなさい。
  そのときの\(\bm{x}\)は\(B\)の最小固有値対応するノルムが\(1\)の
  固有ベクトルであることを示しなさい。
\end{enumerate}

\subsection{回答}

\begin{enumerate}
  \item 行列\(B\)が対称行列であることを示すために、\(B = B^{T}\)を示す。
  \begin{equation}\label{equation:3_1_a}
    B^{T}
    = \left(A^{T} A\right)^{T}
    = A^{T} \left(A^{T}\right)^{T}
    = A^{T} A = B
  \end{equation}
  よって、題意を示せた。\\

  \item 行列\(B\)のすべての固有値\(\lambda\)が0以上であることを示すために、
  すべての零ベクトルでない\(\bm{w} \in R^{n}\)に対して、
  \(\bm{w}^{T} B \bm{w} \geq 0\)を示す。
  \begin{equation}\label{equation:3_2_a}
    \bm{w}^{T} B \bm{w}
    = \bm{w}^{T} A^{T} A \bm{w}
    = \left(A \bm{w}\right)^{T} A \bm{w}
    = \|A \bm{w}\|^{2}
    \geq 0
  \end{equation}
  よって、題意を示せた。\\

  \item \(\bm{y} = A \bm{x}\)より、
  \begin{equation}\label{equation:3_3_a_1}
    \|\bm{y}\|^{2}
    = \|A \bm{x}\|^{2}
    = \left(A \bm{x}\right)^{T} A \bm{x}
    = \bm{x}^{T} A^{T} A \bm{x}
    = \bm{x}^{T} B \bm{x}
  \end{equation}
  である。
  \(\bm{x}^{T} B \bm{x} \)が最大・最小のとき、
  \(\|\bm{x}\|^{2} = 1\)という拘束条件で、
  最大固有値・最小固有値を求めるため、
  \(\bm{x}^{T} B \bm{x}\)を最大化・最小化する。\\
  ラグランジュの未定乗数\(\lambda\)を導入して、
  汎関数\(L\left(\bm{x}, B\right)\)を定義すると、
  \begin{equation}\label{equation:3_3_a_2}
    L\left(\bm{x}, \lambda\right) = \bm{x}^{T} B \bm{x} - \lambda \left(\bm{x}^{T} \bm{x} - 1\right)
  \end{equation}
  である。\\
  \(\frac{\partial L}{\partial \bm{x}}\)は、
  \begin{align}\label{equation:3_3_a_3}
    \frac{\partial L}{\partial \bm{x}}
    &= \frac{\partial}{\partial \bm{x}} \bm{x}^{T} B \bm{x} - \frac{\partial}{\partial \bm{x}} \lambda \left(\bm{x}^{T} \bm{x} - 1\right) \notag\\
    &= 2 \bm{x}^{T} B - 2 \lambda \bm{x}^{T}
  \end{align}
  \(\frac{\partial L}{\partial \bm{x}} = 0\)を求めると、
  \begin{align}\label{equation:3_3_a_4}
    & 2 \bm{x}^{T} B - 2 \lambda \bm{x}^{T} = 0 \notag\\
    \Leftrightarrow & \bm{x}^{T} B = \lambda \bm{x}^{T} \notag\\
    \Leftrightarrow & B \bm{x} = \lambda \bm{x}
  \end{align}
  よって、未定乗数\(\lambda\)が行列\(B\)の固有値であり、
  \(\bm{x}\)がその固有ベクトルであることを示せた。
  また、固有ベクトル\(\bm{x}\)は\(B \bm{x} = \lambda \bm{x}\)より、
  \begin{equation}\label{equation:3_3_a_5}
    \bm{x}^{T} B \bm{x} =  \bm{x}^{T} \lambda \bm{x} = \lambda \|\bm{x}\|^{2}
  \end{equation}
  であり、\(\|\bm{x}\|^{2} = 1\)のときのみ、\(\bm{x}^{T} B \bm{x} = \lambda\)が成り立つ。
  よって、題意が示せた。\\


  % \item \(\|\bm{x}\|^{2} = 1\)より、
  % ベクトル\(\bm{x} \in R^{n}\)は零ベクトルでない。
  % (\ref{equation:3_2_a})式より、
  % \begin{equation}\label{equation:3_3_a_1}
  %   \|\bm{y}\|^{2}
  %   = \|A \bm{x}\|^{2}
  %   = \|A\|^{2} \|\bm{x}\|^{2}
  %   \geq 0
  % \end{equation}
  % \(\|\bm{x}\|^{2} = 1\)なので、
  % \begin{equation}
  %   \|\bm{y}\|^{2}
  %   = \|A\|^{2}
  %   \geq 0
  % \end{equation}
  % よって、\(\|\bm{x}\|^{2} = 1\)という拘束条件のもとで、
  % \(\|\bm{y}\|^{2}\)の最大値は行列\(B\)の
  % 最大固有値に等しいことを示せた。
\end{enumerate}

\newpage

\section{チェビシェフ多項式の零点を用いた選点法に基づく多項式近似問題} \label{section:4}

\subsection{問題}

チェビシェフ多項式の零点を用いた選点法に基づく多項式近似問題について考える。
近似をする関数\(f\left(x\right) \in L_{2\omega} \left[-1,1\right]\)
は以下の規則に従って与えられる。ただし、重み関数\(\omega \left(x\right)\)は、
\(\omega \left(x\right) = \frac{1}{\sqrt{1 - x^2}}\)である。

\begin{itemize}
  \item 自分の学生番号の下2桁を3で割って余りが0の学生は\(f\left(x\right) = e^{\frac{1 - x^2}{2}}\)
  \item 自分の学生番号の下2桁を3で割って余りが1の学生は\(f\left(x\right) = \log \left(x^2 + 2\right)\)
  \item 自分の学生番号の下2桁を3で割って余りが2の学生は\(f\left(x\right) = \frac{1}{x^4 + 1}\)
\end{itemize}

\begin{enumerate}
  \item 与えられた\(f\left(x\right)\)を5次のチェビチェフ多項式の零点を用いた
  選点法により決定される多項式\(\hat{f}\left(x\right)\)で近似しなさい。
  \item 近似誤差を\(e_c \left(x\right) = f\left(x\right) - \hat{f}\left(x\right)\)
  と定義する。\(e_c\)の\(L_{2\omega}\)ノルムを求めなさい。
  さらに、\(e_c\left(x\right)\)のグラフ\(-1 \leq x \leq 1\)を描き、
  区間\(-1 \leq x \leq 1\)における近似誤差について考察しなさい。

\end{enumerate}

\subsection{回答}

学生番号が「52121322」であるので、\(f\left(x\right) = \log \left(x^{2} + 2\right)\)を近似をする関数に選ぶ。

\begin{enumerate}
  \item チェビシェフ多項式は
  \begin{equation}\label{equation:4_1_a_1}
    T_{n} \left(x\right) = \frac{\left(-1\right)^{n}}{\left(2n -1\right)!!}
    \sqrt{1 - x^{2}} \frac{d^{n}}{dx^{n}} \left(1 - x^{2}\right)^{n - \frac{1}{2}}
  \end{equation}
  であり、
  \begin{align}\label{equation:4_1_a_2}
    & T_{0}\left(x\right) = 1 \notag \\
    & T_{1}\left(x\right) = x \notag \\
    & T_{n + 1}\left(x\right)
    = 2 x T_{n} \left(x\right) - T_{n - 1}\left(x\right)
  \end{align}
  という漸化式を満足する。\\
  この漸化式を用いて、5次までのチェビシェフ多項式を作る。
  \begin{align}\label{equation:4_1_a_3}
    & T_{0} = 1 \notag \\
    & T_{1} = x \notag \\
    & T_{2} = 2  x^{2} - 1 \notag \\
    & T_{3} = 4  x^{3} - 3  x \notag \\
    & T_{4} = 8  x^{4} - 8  x^{2} + 1 \notag \\
    & T_{5} = 16 x^{5} - 20 x^{3} + 5x
  \end{align}
  このチェビシェフ多項式から、\(L_{2 \omega} \left[-1,1\right]\)の完全正規多項式
  \(\left\{\bm{u}\right\}^{\infty}_{n = 0}\)の\(n = 0, 1, 2, 3, 4, 5\)を作る。
  重み関数\(\omega \left(x\right)\)は
  \(\omega \left(x\right) = \frac{1}{\sqrt{1 - x^{2}}}\)とする。
  \begin{align}\label{equation:4_1_a_4}
    & \bm{u}_{0} \left(x\right) = \frac{1}{\sqrt{\pi}} T_{0} \left(x\right) = \frac{1}{\sqrt{\pi}} \notag \\
    & \bm{u}_{1} \left(x\right) = \sqrt{\frac{2}{\pi}} T_{1} \left(x\right) = \sqrt{\frac{2}{\pi}} x \notag \\
    & \bm{u}_{2} \left(x\right) = \sqrt{\frac{2}{\pi}} T_{2} \left(x\right) = \sqrt{\frac{2}{\pi}} \left(2 x^{2} - 1\right) \notag \\
    & \bm{u}_{3} \left(x\right) = \sqrt{\frac{2}{\pi}} T_{3} \left(x\right) = \sqrt{\frac{2}{\pi}} \left(4 x^{3} - 3 x\right) \notag \\
    & \bm{u}_{4} \left(x\right) = \sqrt{\frac{2}{\pi}} T_{4} \left(x\right) = \sqrt{\frac{2}{\pi}} \left(8  x^{4} - 8  x^{2} + 1\right) \notag \\
    & \bm{u}_{5} \left(x\right) = \sqrt{\frac{2}{\pi}} T_{5} \left(x\right) = \sqrt{\frac{2}{\pi}} \left(16 x^{5} - 20 x^{3} + 5x\right)
  \end{align}
  完全正規多項式\(\left\{\bm{u}\right\}^{\infty}_{n = 0}\)の零点
  \(z_{0}, z_{1}, z_{2}, z_{3}, z_{4}\)を数式処理ソフトウェア\(maxima\)を用いて求める。
  出力結果より、零点\(z_{0}, z_{1}, z_{2}, z_{3}, z_{4}\)は
  \begin{align}\label{equation:4_1_a_5}
    z_{0} &= -\frac{\sqrt{\sqrt{5}+5}}{{{2}^{\frac{3}{2}}}} = -0.9510565162951534 \notag \\
    z_{1} &=  \frac{\sqrt{\sqrt{5}+5}}{{{2}^{\frac{3}{2}}}} =  0.9510565162951534 \notag \\
    z_{2} &= -\frac{\sqrt{5-\sqrt{5}}}{{{2}^{\frac{3}{2}}}} = -0.5877852522924729 \notag \\
    z_{3} &=  \frac{\sqrt{5-\sqrt{5}}}{{{2}^{\frac{3}{2}}}} =  0.5877852522924729 \notag \\
    z_{4} &=  0.0
  \end{align}
  である。\\
  求めた零点\(z_{0}, z_{1}, z_{2}, z_{3}, z_{4}\)に対して、
  \begin{equation}\label{equation:4_1_a_6}
    e\left(z_{i}\right) = 0 \quad \left(i = 0,1,2,\cdots,n\right)
  \end{equation}
  を満足する、すなわち、
  \begin{equation}\label{equation:4_1_a_7}
    \sum^{n}_{k=0} a_{k} \bm{u}_{k} \left(z_{i}\right) = f\left(z_{i}\right) \quad
    \left(i = 0,1,2,\cdots,n\right)
  \end{equation}
  を満足するように未定の定数\(a_{1}, a_{2}, \cdots, a_{n}\)を決定する。
  (\ref{equation:4_1_a_7})式を連立方程式に変換すると、
  \begin{equation}\label{equation:4_1_a_8}
    \begin{cases}
      a_{0} \bm{u}_{0}(z_{0}) + a_{1} \bm{u}_{1}(z_{0}) + \cdots + a_{n} \bm{u}_{n}(z_{0}) = f(z_{0}) \\
      a_{0} \bm{u}_{0}(z_{1}) + a_{1} \bm{u}_{1}(z_{1}) + \cdots + a_{n} \bm{u}_{n}(z_{1}) = f(z_{1}) \\
      \quad\quad\quad\quad\quad\quad\quad\quad\quad\quad \vdots \\
      a_{0} \bm{u}_{0}(z_{n}) + a_{1} \bm{u}_{1}(z_{n}) + \cdots + a_{n} \bm{u}_{n}(z_{n}) = f(z_{n}) \\
    \end{cases}
  \end{equation}
  ここで、この連立方程式を行列に置き換える。
  \begin{align}\label{equation:4_1_a_9}
    A = \begin{pmatrix}
      \bm{u}_{0}(z_{0}) & \bm{u}_{1}(z_{0}) & \cdots & \bm{u}_{n}(z_{0}) \\
      \bm{u}_{0}(z_{1}) & \bm{u}_{1}(z_{1}) & \cdots & \bm{u}_{n}(z_{1}) \\
      \vdots            & \vdots            & \ddots & \vdots            \\
      \bm{u}_{0}(z_{n}) & \bm{u}_{1}(z_{n}) & \cdots & \bm{u}_{n}(z_{n})
    \end{pmatrix}
    , & &
    \bm{x} = \begin{pmatrix}
      a_{0} \\ a_{1} \\ \vdots \\ a_{n}
    \end{pmatrix}
    , & &
    \bm{b} = \begin{pmatrix}
      f(z_0) \\ f(z_1) \\ \vdots \\ f(z_n)
    \end{pmatrix}
  \end{align}
  このとき、(\ref{equation:4_1_a_7})式は
  \begin{equation}\label{equation:4_1_a_10}
    A \bm{x} = \bm{b}
  \end{equation}
  と表せる。
  \(\bm{u}_{0}, \bm{u}_{1}, \bm{u}_{n}\)は区間\(\left[-1, 1\right]\)
  で互いに線形独立なので、\(|A| \neq 0\)である。
  つまり、(\ref{equation:4_1_a_10})式の連立方程式\(A \bm{x} = \bm{b}\)
  は唯一の解\(\bm{x} = A^{-1} \bm{b}\)を持つ。\\
  \(\bm{x}\)を求めるため、\(maxima\)で行列\(A\)とベクトル\(\bm{b}\)の
  値を算出する。

  \begin{equation}
    {
      \tiny
      A = \begin{pmatrix}
        0.5641895835477563 & -0.7588333108028616 & 0.6455021692389087 & -0.4689847778717805 & 0.2465598888374751\\
        0.5641895835477563 & 0.7588333108028616 & 0.6455021692389087 & 0.4689847778717805 & 0.2465598888374751\\
        0.5641895835477563 & -0.4689847778717812 & -0.2465598888374768 & 0.7588333108028621 & -0.6455021692389088\\
        0.5641895835477563 & 0.4689847778717812 & -0.2465598888374768 & -0.7588333108028621 & -0.6455021692389088\\
        0.5641895835477563 & 0.0 & -0.7978845608028654 & 0.0 & 0.7978845608028654
      \end{pmatrix}
    }
  \end{equation}

  \begin{equation}
    {
      \tiny
      \bm{b} = \begin{pmatrix}
        1.066264183971209 & 1.066264183971209 & 0.8524949760181532 & 0.8524949760181532 & 0.6931471805599453
      \end{pmatrix}
    }
  \end{equation}

  これより、\(\bm{x} = A^{-1} \bm{b}\)を\(maxima\)で求める。
  \begin{equation}
    \bm{x} = \begin{pmatrix}
      1.606079102718907\\
      0.0\\
      0.2532865853005147\\
      -1.050019021388809 \cdot {{10}^{-16}}\\
      -0.01365167871787071
    \end{pmatrix}
    \simeq \begin{pmatrix}
      1.606079102718907\\
      0.0\\
      0.2532865853005147\\
      0.0\\
      -0.01365167871787071
    \end{pmatrix}
  \end{equation}
  よって、近似関数は
  \begin{equation}
    \begin{split}
      \hat{f}\left(x\right)
      =& 1.606079102718907    \cdot \bm{u}_{0}\left(x\right) + 0.0 \cdot \bm{u}_{1}\left(x\right)\\
      &+ 0.2532865853005147   \cdot \bm{u}_{2}\left(x\right) + 0.0 \cdot \bm{u}_{3}\left(x\right)
       - 0.01365167871787071  \cdot \bm{u}_{4}\left(x\right)\\
      =& 1.606079102718907    \cdot \bm{u}_{0}\left(x\right)
       + 0.2532865853005147   \cdot \bm{u}_{2}\left(x\right)\\
      &- 0.01365167871787071  \cdot \bm{u}_{4}\left(x\right)\\
      =& -0.08883698847953755 {{x}^{4}}+0.4982243501897492 {{x}^{2}}+0.6931471805599454
    \end{split}
  \end{equation}

  \begin{figure}[htbp]
    \centering
    \includegraphics[width=80mm]{./image/kadai1.png}
    \caption{ターゲット関数と近似関数の比較}
    \label{figure:4_1_a}
  \end{figure}

  \item 近似誤差を
  \begin{equation}
    e_{c}\left(x\right) = f\left(x\right) - \hat{f}\left(x\right)
  \end{equation}
  と定義すると、\(e_{c}\left(x\right)\)の\(L_{2\omega}\)ノルムは
  \begin{align}
      \|e_{c}\left(x\right)\|
      &= \sqrt{\left<e_{c}\left(x\right), e_{c}\left(x\right)\right>}\notag\\
      &= \sqrt{\int^{1}_{-1} \omega\left(x\right) e_{c}\left(x\right)^{2} dx}
      &= \sqrt{\int^{1}_{-1} \frac{1}{\sqrt{1 - x^{2}}} \left(f\left(x\right) - \hat{f}\left(x\right) \right)^{2} dx}
  \end{align}
  この式を\(maxima\)で算出する。
  \begin{equation}
    \|e_{c}\left(x\right)\| = 0.001221761613102146
  \end{equation}

  また、\(e_c\left(x\right)\)のグラフ\(-1 \leq x \leq 1\)を描く。

  \begin{figure}[htbp]
    \centering
    \includegraphics[width=80mm]{./image/kadai4_2.png}
    \caption{区間[-1,1]における近似誤差}
    \label{figure:4_2_a}
  \end{figure}

  考察として、区間\(-1 \leq x \leq 1\)における近似誤差について、
  \(x\)の値が\(0\)から離れるにつれて、誤差が大きくなる。
  しかし、区間\(-1 < x < 1\)では大きくて\(0.001\)ほどの誤差しか発生しないので、
  十分に誤差が小さいといえる。

%   \newpage

%   \begin{lstlisting}[caption={4-(1)で使用したmaximaコード},label={code:kadai4_1}]
% /* 完全正規多項式の定義 */
% u0(x):=float(1/sqrt(%pi))$
% u1(x):=float(sqrt(2/%pi)*x)$
% u2(x):=float(sqrt(2/%pi)*(2*x^2-1))$
% u3(x):=float(sqrt(2/%pi)*(4*x^3-3*x))$
% u4(x):=float(sqrt(2/%pi)*(8*x^4-8*x^2+1))$
% u5(x):=sqrt(2/%pi)*(16*x^5-20*x^3+5*x)$
% /* 近似する関数の定義 */
% fa(x):=float(log(x^2+2))$

% /* 零点の算出 */
% z:solve(u5(t),t)$
% for i thru 5 do (
%     print("z[", i, "] = ", rhs(z[i]), " = ", float(rhs(z[i]))),
%     z[i]:float(rhs(z[i]))
% )$

% /* 行列Aの定義 */
% A:matrix(
%     [u0(z[1]),u1(z[1]),u2(z[1]),u3(z[1]),u4(z[1])],
%     [u0(z[2]),u1(z[2]),u2(z[2]),u3(z[2]),u4(z[2])],
%     [u0(z[3]),u1(z[3]),u2(z[3]),u3(z[3]),u4(z[3])],
%     [u0(z[4]),u1(z[4]),u2(z[4]),u3(z[4]),u4(z[4])],
%     [u0(z[5]),u1(z[5]),u2(z[5]),u3(z[5]),u4(z[5])]
% )$
% /* ベクトルbの定義 */
% b:matrix(
%     [fa(z[1]),fa(z[2]),fa(z[3]),fa(z[4]),fa(z[5])]
% )$

% /* Ax=b の解xの算出 */
% c:invert(A).b$
% /* 近似多項式の定義 */
% fb(x):=1.606079102718907*u0(x)+0.2532865853005147*u2(x)-0.01365167871787071*u4(x)$

% /* 行列A、ベクトルb、解xの表示 */
% print("A = ", A)$
% print("b = ", b)$
% print("x = ", c)$
% print("fb(x) = " , fb(x), " = ", expand(fb(x)))$

% /* 指定した関数と近似多項式を表示 */
% plot2d([fa(x),fb(x)], [x,-1.0,1.0],
%     [legend, "Target function", "Approximate function"],
%     [xlabel, "x"],
%     [ylabel, "f(x)"])$
%   \end{lstlisting}

%   \begin{lstlisting}[caption={4-(2)で使用したmaximaコード},label={code:kadai4_2}]
% /* 誤差を算出 */
% ec(x):=fa(x)-fb(x)$
% omega(x):=1.0/sqrt(1.0-x^2)$
% sq:quad_qags(omega(x)*(ec(x)^2),x,-1.0,1.0)$
% nr:sqrt(sq[1])$

% /* 誤差の表示 */
% print("norm = ", nr)$

% /* 誤差をグラフ化 */
% plot2d([ec(x)], [x,-1.0,1.0],
%     [xlabel, "x"],
%     [ylabel, "e_c(x)"])$
%   \end{lstlisting}

\end{enumerate}
